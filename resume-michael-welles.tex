%% MLW Resume
%% Derived from `template_en.tex'.
%% And jfb.tex
%% Copyright 2006-2008 Xavier Danaux (xdanaux@gmail.com).
%
% This work may be distributed and/or modified under the
% conditions of the LaTeX Project Public License version 1.3c,
% available at http://www.latex-project.org/lppl/.

\documentclass[letterpaper]{moderncv}

%\moderncvstyle{classic} % CV theme - options include: 'casual' (default), 'classic', 'oldstyle' and 'banking'
%\moderncvcolor{green} % CV color - options include: 'blue' (default), 'orange', 'green', 'red', 'purple', 'grey' and 'black'	

\moderncvtheme[blue]{classic} 
\usepackage[utf8]{inputenc} 

\usepackage[scale=0.8]{geometry}% Reduce document margins
%\setlength{\hintscolumnwidth}{3cm} % Uncomment to change the width of the dates column
%\setlength{\makecvtitlenamewidth}{10cm} % For the 'classic' style, uncomment to adjust the width of the space allocated to your name

%\AtBeginDocument{\recomputelengths} 

% personal data
\firstname{Michael}
\familyname{Welles}
% All information in this block is optional, comment out any lines you don't need
%\title{Curriculum Vitae}
\address{5657 NY-97}{Narrowsburg NY 12764}
\phone{917-586-9218}
\email{mlwelles@gmail.com}

%\photo[70pt][0.4pt]{Pictures/mlw-portrait.jpg} % The first bracket is the picture height, the second is the thickness of the frame around the picture (0pt for no frame)
%----------------------------------------------------------------------------------------

\begin{document}

%\maketitle -- old, dont know what it does (mlw)
%----------------------------------------------------------------------------------------
%	CURRICULUM VITAE
%----------------------------------------------------------------------------------------
\makecvtitle % Print the CV title

\section{Overview}

% Overview
% Reworked for clarity and concision.
\cvline{}{I am a software engineer and technology leader with decades of experience building products and leading teams under tight time and quality constraints.  I am also a strong advocate for efficient processes, clear communication and best practices in software development.  My experience ranges from writing firmware for embedded devices to architecting distributed systems that process complex encoding workflows across thousands of machines.}
\section{Experience}
\cventry{2024-2025}{Principal Engineer}{Istari Digital}{New York}{NY}{Istari Digital focuses on interconnected digital systems where secure, shareable digital threads facilitate the development of cyber‑physical systems.  The platform enables zero‑trust, zero‑knowledge security while connecting to tools such as Cameo, CATIA, Nastran, OpenFOAM and other computational modeling suites to accelerate digital‑twin development and facilitate secure collaboration.}
% Key contributions at Istari Digital
\cvline{-}{Software architect and technical lead for 'zero-trust and zero-knowledge' core platform rewrite to meet the strict security and threading requirements of defense and aeorspace customers while still providing for frictionless workflow automation, secure collaboration, and trust in the integrety of cryptographically verified and version managed digital assets, preserving a complete digital thread not just of the changes to the assets but to their lineage -- what versions of what assets were used as inputs to produce what versions which other assets, as well what was used to do and by whom.    All without the backend ever accessing any of sensitive contents of the files themselves.}
\cvline{-}{Implemented DoD‑compliant control tagging to ensure customers retain data sovereignty.  This provided strict controls and audit trails while enabling frictionless sharing of models and artifacts among individual engineers, teams and organizations.}
\cvline{-}{Delivered all major program milestones on time and successfully relaunched the product for commercial and government clients.}
\cvline{-}{Led two teams designing and implementing a secure backend registry service in Python and the SDKs for accessing it.  The cryptographic core was written in Rust and compiled into Python bindings and WebAssembly, forming the foundation for the Python and TypeScript libraries used by automation agents and the web application.}
\cventry{2020-2023}{Director of Software Development}{Raytheon Technologies}{New York}{NY}{Rejoined what was formerly the UTC Digital Accelerator (DX)—reorganized post‑merger as Enterprise Data Services (EDX)—to build a next‑generation data platform for Raytheon aerospace applications.}
% Key contributions at Raytheon Technologies
\cvline{-}{Technical lead for pathfinder initiatives developing streaming flight telemetry pipelines for multiple models of Pratt \& Whitney commercial jet engines.}
\cvline{-}{Designed systems where, upon gate arrival, aircraft connect to ground stations to upload telemetry from thousands of sensors monitoring engines and control surfaces.}
\cvline{-}{Within minutes, processed datasets and applied algorithmic fault detection, machine‑learning failure prediction and anomaly detection models.}
\cvline{-}{Triggered alerts with responses proportional to severity and confidence—from grounding aircraft for emergency maintenance to scheduling inspection at subsequent destinations.}
\cvline{-}{Maintained comprehensive audit trails to trace the lineage of every output field back to specific code revisions or model versions, enabling reproducibility.}
\cvline{-}{Managed model‑training workflows with rigorous versioning; captured code revisions, training datasets and hyperparameters so results could be reproduced.}
\cvline{-}{Technologies included Databricks, Spark, Python, SparkML, scikit‑learn and Pandas.}
\cvline{-}{Led a team of 14 developers and contractors split across three agile project teams.}
\cvline{-}{Supervised creation of onboarding resources for more than 40 teams, publishing guidelines, standards, best practices and reference project templates.}
\cvline{-}{Oversaw teams porting and optimizing existing internal data science and machine‑learning libraries and tools to the new platform.}
\cventry{2020-2021}{Head of Technology}{Dayforward}{New York}{NY}{Head of technology and development lead for a life‑insurance startup. Led a small team that designed and built the company’s algorithmic underwriting and policy‑management platform.  The platform comprised containerized Go gRPC microservices deployed on Kubernetes and exposed via a federated GraphQL API to a Vue.js frontend.  Initial development was completed in under ten months, and the platform launched the same day the company received regulatory approval.}

\cventry{2019}{Director of Software Engineering}{UTC Aerospace Systems}{Brooklyn}{NY}{Managed a team of 17 engineers at the UTC Digital Accelerator in Brooklyn.  Oversaw multiple project teams—from IoT sensors and mobile apps for industrial refrigeration to standardized design systems and developer tools.  As the frontend engineering director, led efforts to normalize, document and evangelize engineering processes, standards and best practices.}

\cventry{2018}{Independent Contractor}{Barking For Centuries, LLC}{New York}{NY}{Served as chief technologist for an early‑stage startup and led a team of senior contractors to build a vertical product search engine.}
% Key contributions at Barking For Centuries
\cvline{-}{Designed and implemented an ETL pipeline built around custom machine‑learning infrastructure for product identification and extraction.  The pipeline was written in Python with NLP components using spaCy for classification and entity extraction, along with NLTK and scikit‑learn.}
\cvline{-}{Developed product image recognition and classification models on AWS SageMaker, with preprocessing and color analysis using OpenCV and scikit‑image.}
\cvline{-}{Used DynamoDB and AWS SQS/SNS for data flow and message storage.}
\cvline{-}{Built the search API in Go against Elasticsearch indexes and developed the web frontend in React.js.}
\cvline{-}{Architected the system as microservices and discrete transformation steps, each packaged as a Docker image and deployed via CI/CD to a Kubernetes cluster that automatically scaled both pods and worker nodes.  The cluster itself was built and managed with Terraform.}



\cventry{2017}{Director of Engineering (Mobile)}{MediData}{New York}{NY}{Led engineering teams responsible for the Patient Cloud platform, which collects clinical trial data directly from patients and clinicians via mobile devices and wearable sensors.}
% Key contributions at MediData
\cvline{-}{Managed products including ePRO (iOS and Android app for patient‑reported outcomes), Patient Cloud (iOS tablet app for clinician‑reported outcomes), AppConnect (native SDK for third‑party developers), Sensor Link (platform for ingesting data from wearables) and the supporting backend.}
\cvline{-}{Launched two major mobile product initiatives and migrated all native development to Swift and Kotlin.}
\cvline{-}{Built compliance dashboards for data reporting and initiated migration of data collection and analysis to a scalable real‑time stream‑processing framework (Apache Flink).}
\cvline{-}{Instituted organizational and process improvements that increased average team velocity by 2.5×.}

\cventry{2013 - 2016}{Director of Mobile}{Huge}{Brooklyn}{NY}{Started as Principal Architect and was later promoted to Director of Mobile.}
% Key contributions at Huge
\cvline{-}{Led a cross‑functional team of more than 20 iOS, Android and backend engineers, QA analysts, designers and product managers.  Evangelized agile best practices, continuous integration and continuous delivery.}
\cvline{-}{As Principal Architect, sponsored new technology investigations and initiatives such as interior way‑finding, Leap Motion, Arduino and embedded system prototypes, and championed internal outreach through an engineering blog, meetups and open‑source efforts.}
\cvline{-}{Directed notable client projects including smart Bluetooth audio/video accessories supporting live video streaming and voice commands, a companion application for an AAA game publisher that scanned players’ likenesses into avatars, and numerous B2B and B2C mobile commerce applications.}

\cventry{2011 - 2013}{Manager of Mobile Technology}{Consumer Reports}{Yonkers}{NY}{
Founded the mobile applications and new media group and built an in‑house team responsible for mobile application development.  Defined mobile product strategies and led technical execution.  Launched the flagship ratings application and managed external vendors maintaining a portfolio of legacy apps.
}

%\cventry{April 2011– July 2011}{Consultant}{Poll Everywhere}{San Francisco}{CA}{Developed OSX desktop version of presenter application, for use in displaying dynamically updating poll results results and charts during presentations}

%\cventry{April 2008– April 2011}{Senior Software Engineer - iTunes Store
\cventry{2008 - 2011}{Senior Software Engineer – iTunes Store Video Workflow Group}{Apple}{Cupertino}{CA}{
Senior engineer on a team of five responsible for encoding and assembling all iTunes video media.  Owned the encoding toolchain used by the processing cluster and specified deliverable media formats, developed test suites to validate them and created reference media for hardware compliance testing.  Contributed to continuous improvement and day‑to‑day operations, business production and the engineering of visual and audio quality of Store media.  Led two major rewrites of the video workflow: one for the HDTV launch and another for international video and television.
}



%\cventry{August 2007 – March 2008}{Senior Software Architect}{The New
\cventry{ 2007 - 2008}{Senior Software Architect}{The New York Times}{New York}{NY}{
Led the incorporation of continuous integration into the development and release processes of NYTimes.com.  Created tools, procedures and processes for testing and packaging software through the production pipeline and authored automation tools.  Supervised overall architectural design for the development of a new content management system.
}

 
\cventry{2006 - 2007}{Founder, Senior Partner}{Bangstate}{New York}{NY}{
Resumed role as senior partner focusing on web applications using Ruby on Rails and other open‑source technologies.  Emphasized best practices, test‑driven development, short milestones, iterative development and close client interaction.  Served as acting CTO for haystack.com, a music discovery and streaming social network, leading a five‑person design and development team.
  }

%\subsection{The Associated Press}
%\cventry{March  2004 – Sep 2006}{Systems Architect / Lead Developer}{The Associated Press}{New York}{NY}{Design, architecture of highly trafficked and
\cventry{ 2004 - 2006}{Systems Architect/Lead Developer}{The Associated Press}{New York}{NY}{
Designed and developed high‑traffic, dynamic systems for distributing, processing and displaying multimedia news content.  Led a team of seven developers, QA staff and system engineers.  Notable project: AP Hosted Elections, which gathered, processed and presented up‑to‑the‑minute results for the 2004 U.S. presidential election and served as the sole source of data for all major U.S. news organizations.
 }
 
%\subsection{Bangstate Inc.}
%\cventry{September 1998  – March 2004}{Founder, Senior Partner}{Bangstate Inc.}{New York}{NY}{Founded and managed a consultancy of  
\cventry{1998 - 2004}{Founder, Senior Partner}{Bangstate}{New York}{NY}{
Founded and managed a consultancy of five principal partners and ten additional developers, designers and administrators.  Oversaw the business and delivered projects for clients including The Associated Press, American Bar Association, Atlantic Records, Forbes Magazine, CIR/SEIU and Time Inc.  Notable projects included TNEWS, an Internet distribution platform for the Associated Press that used standards‑based protocols (NNTP, IPTC NITF XML) to distribute content previously available only via proprietary satellite feeds, and handheld applications for the Military Family Research Institute at Purdue University.
}

%\cventry{August 1996 - January 1997}{Member, Technical Staff}{SOS Corp}{New York}{NY}{}
\cventry{1996 - 1997}{Member, Technical Staff}{SOS Corp}{New York}{NY}{}

%\cventry{September 1995 - August 1996}{System Administrator -
\cventry{1995 - 1996}{System Administrator -
  Distributed Resources Management}{PaineWebber}{New York}{NY}{}

%\cventry{June 1995 - September 1995}{Consultant/Programmer}{onShore,
\cventry{1995}{Consultant/Programmer}{onShore}{Chicago}{IL}{} 
%  \cventry{September 1994 - June 1995
  \cventry{1994}{Jr. Programmer/Analyst}{The Social Science and Public Policy
  Computing Center at the University of Chicago}{Chicago}{IL}{}

\section{Education}

\cventry{1990 - 1994}{Bachelor of Arts in History}{The University of Chicago}{Chicago, IL}{}{}

\section{OSS Software} 
\cvline{}{OSS contributions not otherwise noted in work history;}
\cvline{}{
Author, BeaconScanner, desktop utility for OSX for detecting and managing iBeacons.  http://tinyurl.com/j6rpfbc  (2014)
}

\end{document}

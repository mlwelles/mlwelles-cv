%% MLW Resume
%% Derived from `template_en.tex'.
%% And jfb.tex
%% Copyright 2006-2008 Xavier Danaux (xdanaux@gmail.com).
%
% This work may be distributed and/or modified under the
% conditions of the LaTeX Project Public License version 1.3c,
% available at http://www.latex-project.org/lppl/.

\documentclass[letterpaper]{moderncv}

\moderncvstyle{classic} % CV theme - options include: 'casual' (default), 'classic', 'oldstyle' and 'banking'
\moderncvcolor{blue} % CV color - options include: 'blue' (default), 'orange', 'green', 'red', 'purple', 'grey' and 'black'	

%\moderncvtheme[blue]{classic} 
\usepackage[utf8]{inputenc} 

\usepackage[scale=0.8]{geometry}% Reduce document margins
%\setlength{\hintscolumnwidth}{3cm} % Uncomment to change the width of the dates column
%\setlength{\makecvtitlenamewidth}{10cm} % For the 'classic' style, uncomment to adjust the width of the space allocated to your name

%\AtBeginDocument{\recomputelengths} 

% personal data
\firstname{Michael}
\familyname{Welles}
% All information in this block is optional, comment out any lines you don't need
%\title{Curriculum Vitae}
\address{55 E. 76th St. Apt 13}{New York, NY 10021}
\phone{917-586-9218}
\email{mlwelles@gmail.com}

%\photo[70pt][0.4pt]{Pictures/mlw-portrait.jpg} % The first bracket is the picture height, the second is the thickness of the frame around the picture (0pt for no frame)
%----------------------------------------------------------------------------------------

\begin{document}

%\maketitle -- old, dont know what it does (mlw)
%----------------------------------------------------------------------------------------
%	CURRICULUM VITAE
%----------------------------------------------------------------------------------------
\makecvtitle % Print the CV title

\section{Overview}

\cvline{}{Years of experience as a developer and leading technology teams and delivering projects under strict time and quality constraints.   Strong advocate for efficient processes, effective communication, and best practices in software development.   Adept technologist with experience working on everything from from firmware for embedded devices to distributed systems processing complex encoding workflows across thousands of systems.}

\section{Experience}

\cventry{2020-}{Director of Software Development}{Raytheon Technologies}{New York}{NY}{Rejoined what was formerly the UTC Digital Accelerator (DX), now reorganized post-merger as Enterprise Data Services (EDX)  to focus on building a next generation data platform for Raytheon aerospace applications}
\cvline{-}{Technical lead for team developing flight data platform pathfinder initiatives, creating streaming flight telemetry pipelines for multiple models of Pratt \& Whitney commercial jet engines.}
\cvline{}{Upon arrival at terminal gate, an aircraft is connected to a ground station which uploads the  telemetry captured from the thousands of sensors monitoring the engines and control surfaces during the flights.}  
\cvline{}{Within a five minutes the datasets are processed and the results from algorithmic fault detection and ML failure prediction and anomaly detection model  scoring applied.}   
\cvline{}{Alerts are triggered from the results, with actions taken in response varying depending on the severity of the fault or the confidence of the score -  grounding the aircraft  for emergency maintenance, or scheduling a team to meet it at its next or subsequent destination a to perform inspection and repair.}
\cvline{}{A comprehensive  audit trail is kept so that lineage for each field in the records produced can traced back to a specific code revision or model version, and can be recreated.}
\cvline{}{Model training workflows are similarly managed, each new model versioned, and the code revision, training data set, and hyper parameters specified to produce it captured and reproducible.}
\cvline{}{Technologies:  Databricks, Spark, Python, SparkML, SkLearn, Pandas. }
\cvline{-}{Led team 14 of developers and contractors, split between three agile project teams}
\cvline{-}{Supervised creation of developer resources for the 40+ teams onboarding on to the new platform, publishing guidelines, standards, best practices, and reference project template.}
\cvline{-}{Oversaw teams responsible for porting and optimizing existing internal data science and machine learning libraries and tools to new platform.}
\cventry{2020-2021}{Head of Technology}{Dayforward}{New York}{NY}{Head of technology and development lead for life insurance startup. Led small team that designed and built the companies algorithmic underwriting and policy management platform.  Initial development was completed in under ten months and the platform and application were ready to launch the same day the company received regulatory approval to do so.  The platform was implemented as containerized go gRPC microservices deployed on kubernetes and accessed from the accompanying frontend Vue application via a federated GraphQL API.}

\cventry{2019}{Director of Software Engineering}{UTC Aerospace Systems}{Brooklyn}{NY}{Managed team of 17 engineers at DX, the digital accelerator UTC established in Brooklyn NY. Oversaw multiple project teams, ranging from IOT sensors and accompanying mobile applications for managing
industrial refrigeration to standardized design systems with accompanying developer
tools. Frontend engineering director, led efforts to normalize, document and evangelize engineering processes, standards and best practices.}

\cventry{2018}{Independent Contractor}{Barking For Centuries, LLC}{New York}{NY}{Serving as chief technologist for an early stage startup, led team of contracted senior developers to build a category vertical product search engine. The extract / transform / load (ETL) pipeline at the core of the system relies entirely on custom built machine learning infrastructure for product identification and extraction.    The pipeline was written in Python, the NLP components using spaCy for classification and entity extraction, as well as NLTK and scikit-learn for text extraction.  The product image recognition and classification models were built on SageMaker,  with preprocessing and color analysis using OpenCV,  scikit-image, and others.  DynamoDB  and SQS / SNS for the pipeline data flow and message store.   The search API was written in Go,  querying against Elasticsearch indexes.    The web frontend was written in React JS.    All components of the system were designed as either micro-services or discrete data flow transformation steps, each built and deployed as discrete docker images.   These images were deployed automatically via CI/CD as  pods on to a Kubernetes cluster which would automatically scale the the number of each it ran to match demand.  The Kubernetes cluster, was built and managed using Terraform and it, in turn configured so that the number of worker nodes in the cluster also auto scaled to meet demand. }



\cventry{2017}{Director of Engineering (Mobile)}{MediData}{New York}{NY}{
	   Led engineering teams responsible for Patient Cloud platform supporting the collection of clinical trial data directly from patients and clinicians on mobile devices and wearable sensors.   Products include ePRO (the iOS and Android application used for patient reported outcomes), Patient Cloud (an iOS tablet application for clinician reported outcomes), AppConnect (a native SDK for use by third party developers), Sensor Link (a platform for direct ingestion of data from wearable sensors), and the backend platform used by the products. During tenure, launched two new major mobile product initiatives.  Also migrated all native mobile development to Swift and Kotlin. Additionally, build compliance dashboards for data reporting, and launched effort to migrate data collection and analysis on to a scalable real time stream processing framework (Apache Flink).  Introduced  organizational and process improvements that resulted in a 2.5x increase in average team velocity. }

\cventry{2013 - 2016}{Director of Mobile}{Huge}{Brooklyn}{NY}{Started as Principal Architect, promoted to Director of Mobile.  In latter role, led 20+ person cross-functional team of iOS, Android, and backend engineers - as well as QA analysts, designers, and product managers. Institute and evangelize agile best practices, continuous integration, and continuous delivery. In Principal Architect role, sponsored new technology investigations and initiatives (interior way-finding,  Leap Motion, Arduino, and embedded system prototypes), as well as internal outreach initiatives (engineering blog, organizing meetups, open source efforts, etc.). Notable client work include `smart' bluetooth audio/video accessories that support live video streaming and voice commands, among other features, and a companion application for an AAA game publisher that allows players to scan their own likeness to be rendered as an in-game avatar, as well as numerous  BTB and BTC mobile commerce applications.}

\cventry{2011 - 2013}{Manager of Mobile Technology}{Consumer Reports}{Yonkers}{NY}{
Founded mobile applications and new media group. Built in-house team responsible for mobile applications development. Responsibilities included overall mobile product strategies and technical execution. Launched flagship ratings application and managed external vendors in the maintenance of a portfolio of legacy applications.
}

%\cventry{April 2011– July 2011}{Consultant}{Poll Everywhere}{San Francisco}{CA}{Developed OSX desktop version of presenter application, for use in displaying dynamically updating poll results results and charts during presentations}

%\cventry{April 2008– April 2011}{Senior Software Engineer - iTunes Store
\cventry{2008 - 2011}{Senior Software Engineer - iTunes Store
  Video Workflow Group}{Apple}{Cupertino}{CA}{
Senior engineer on a team of five responsible for the encoding and assembly of all iTunes video media. Responsible for the encoding toolchain used by the processing cluster, as well as for the specification of formats of deliverable media, the test suites to validate the formats, and the creation of reference media for use by hardware teams in compliance testing.  Heavily involved in the continuous improvement, day-to-day operations, business production, and engineering of the visual and audio quality of Store media. Led two major rewrites of the video workflow - the first for the launch of the HDTV product and the second for the launch of international video and television.
}



%\cventry{August 2007 – March 2008}{Senior Software Architect}{The New
\cventry{ 2007 - 2008}{Senior Software Architect}{The New York Times}{New York}{NY}{
Led effort to incorporate continuous integration into the development and release processes of NYTimes.com. Created tools, procedures, and processes for the testing and packaging of software through the chain of production. Authored tools to help with process automation. Supervised overall architectural design for the effort to develop a new content management system for news.
}

 
\cventry{2006 - 2007}{Founder, Senior Partner}{Bangstate}{New York}{NY}{
Resumed role as senior partner, concentrating on web application using Ruby on Rails and other open source technologies. Strong focus on best practices, test-driven development, short milestones, iterative development, and close client interaction. Also served as acting CTO for haystack.com, a social networking site focused on music discovery and full album streaming content. In that role, led five person design and development team.
  }

%\subsection{The Associated Press}
%\cventry{March  2004 – Sep 2006}{Systems Architect / Lead Developer}{The Associated Press}{New York}{NY}{Design, architecture of highly trafficked and
\cventry{ 2004 - 2006}{Systems Architect/Lead Developer}{The Associated Press}{New York}{NY}{
Designed and developed highly trafficked and dynamic systems for the distribution, digestion, and display of multimedia news content. Led team of seven developers, QA staff, and system engineers. Notable project was the development of AP Hosted Elections, a system that gathered, processed, and presented the up-to-the minute results of for the 2004 US presidential elections,  which served as the sole source of 2004 presidential election data for all major US news organizations. 
 }
 
%\subsection{Bangstate Inc.}
%\cventry{September 1998  – March 2004}{Founder, Senior Partner}{Bangstate Inc.}{New York}{NY}{Founded and managed a consultancy of  
\cventry{1998 - 2004}{Founder, Senior Partner}{Bangstate}{New York}{NY}{
Founded and managed a consultancy of five principal partners, and ten additional developers, designers, and administrators. Responsible for the business as a whole, as well as delivering individual projects. Clients included organizations such as The Associated Press, The American Bar Association, Atlantic Records, Forbes Magazine, CIR/SEIU, Time, Inc. Notable projects included: TNEWS, the Internet distribution platform of the Associated Press that used standards-based protocols (NNTP, IPTC NITF XML) to distribute content that had previously only been available in proprietary formats over satellite and developing handheld applications for use by Military Family Research Institute at Purdue University.
}

%\cventry{August 1996 - January 1997}{Member, Technical Staff}{SOS Corp}{New York}{NY}{}
\cventry{1996 - 1997}{Member, Technical Staff}{SOS Corp}{New York}{NY}{}

%\cventry{September 1995 - August 1996}{System Administrator -
\cventry{1995 - 1996}{System Administrator -
  Distributed Resources Management}{PaineWebber}{New York}{NY}{}

%\cventry{June 1995 - September 1995}{Consultant/Programmer}{onShore,
\cventry{1995}{Consultant/Programmer}{onShore}{Chicago}{IL}{} 
%  \cventry{September 1994 - June 1995
  \cventry{1994}{Jr. Programmer/Analyst}{The Social Science and Public Policy
  Computing Center at the University of Chicago}{Chicago}{IL}{}

\section{Education}

\cventry{1990 - 1994}{Bachelor of Arts in History}{The University of Chicago}{Chicago, IL}{}{}

\section{OSS Software} 
\cvline{}{OSS contributions not otherwise noted in work history;}
\cvline{}{
Author, BeaconScanner, desktop utility for OSX for detecting and managing iBeacons.  http://tinyurl.com/j6rpfbc  (2014)
}

\end{document}
